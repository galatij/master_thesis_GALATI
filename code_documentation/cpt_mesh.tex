### Explanation of `cpt_mesh` Function  

This MATLAB function generates a **hexahedral finite element mesh** for a 3D domain with a fault surface. The **Lagrange Multiplier method** is used to enforce the contact and Coulomb friction conditions at the fault. Below is a breakdown of the function’s key operations:

---

### **1. Mesh Grid and Element Definition**
- The domain is discretized into **hexahedral elements**, with **2nx elements along the x-direction**. This doubling in the x-direction is essential for handling the fault.
- **Nodes and Elements:**
  - The number of **nodes** in the x-direction is `nnx = 2*(nx+1)`, ensuring a doubled grid around the fault.
  - The number of elements in x, y, and z directions are `ne = 2*nx*ny*nz`.

- **Coordinates of Nodes (`coord`)**:
  - Nodes are assigned **(x, y, z) coordinates**.
  - The x-coordinates are constructed in such a way that **a duplicate layer is created at x = 0**, which is where the fault lies.
  - The original coordinates are stored in `coordOrig` before modifications.

---

### **2. Fault Representation**
- The fault is defined as **a duplicated set of nodes at x = 0**, ensuring that **the mesh is non-conforming across the fault**.
- The elements are **divided into two groups**: 
  - **Regular elements (`topol`)**: Elements that do not touch the fault.
  - **Interface elements (`interf`)**: Elements that lie at the fault interface.

- The **`id` array** marks elements that lie on the fault plane, and those elements are extracted separately.

- **Material assignment (`matID`)**:
  - Elements are split into two regions based on whether their average x-coordinate is negative or positive.

---

### **3. Modification of Fault Nodes**
- The duplicated nodes at `x = 0` (on the fault) are **forced to have the same coordinates** to ensure continuity of the interface.

- **`interf2e` mapping**: This maps interface elements to the two elements they are in contact with. This mapping will be useful when implementing the Lagrange multipliers for enforcing contact conditions.

---

### **4. Boundary Conditions**
- The function applies **Dirichlet boundary conditions** (`dir`, `dirval`) to:
  - Bottom and top surfaces (`z = 0` and `z = Lz`).
  - Left and right surfaces (`x = ±Lx`).
  - Front and back surfaces (`y = ±Ly`).
  
  Many constraints are commented out, allowing for flexibility in different simulations.

---

### **Fault Treatment in the Mesh**
1. **Non-Conforming Mesh Across the Fault**:
   - The fault is **not explicitly meshed as a surface**; instead, a non-matching discretization is used, where elements on either side are **duplicated but do not share nodes**.

2. **Discrete Fracture Model (DFM) Context**:
   - The function follows a **Discrete Fracture Model (DFM)** approach, where the fault is represented as a **discontinuity inside the domain** rather than as a boundary.
   - The displacement field is **allowed to be discontinuous**, which enables slip along the fault.

3. **Mortar-Based Approach**:
   - Since **Lagrange Multipliers** are used for enforcing contact, the formulation is compatible with a **mortar-based approach**. This allows for more accurate enforcement of contact constraints in FEM.

---

### **Summary**
- The mesh **doubles elements** in the x-direction to introduce a **fault at x = 0**.
- Nodes at the fault are **duplicated and mapped** for contact enforcement.
- The mesh is **non-conforming**, allowing slip and separation at the fault.
- The function **prepares the data structures** needed for contact handling using **Lagrange Multipliers**.

Would you like further clarification on any specific part? 🚀